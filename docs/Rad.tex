\documentclass[]{foi} 
% zakomentirati za pisanje rada na engleskom jeziku
% comment the above line if writing in English

% \documentclass[english]{foi} 
% odkomentirati za pisanje rada na engleskom jeziku
% uncomment the above line if writing in English

\usepackage[utf8]{inputenc}
\usepackage{lipsum}

\vrstaRada{\diplomski}
% \zavrsni ili \diplomski ili \seminar ili \projekt
% type of the paper:
% \seminar is a seminar/term paper
% \projekt is a project

\title{Naslov rada -- \LaTeX\ Predložak}
\predmet{}
% ostaviti prazno ako \vrstaRada nije \projekt ili \seminar
% \predmetBP ili \predmetDP ili \predmetTBP ili \predmetVAS ili \predmetUUI
% leave it empty if \vrstaRada is not a \projekt or \seminar
% \predmetBP - Databases 1
% \predmetVAS - Multiagent Systems

\author{Barica} 
% ime i prezime studenta/studentice
% name and surname of the author
\spolStudenta{\zensko} 
% \zensko ili \musko
% student's gender for grammar purposes: \zensko = F or \musko = M

\mentor{Bogdan Okreša Đurić}
% ime i prezime mentora
% name and surname of the mentor
\spolMentora{\musko} 
% \zensko ili \musko
% mentor's gender: \zensko = F or \musko = M
\titulaProfesora{doc.~dr.~sc.}
% HR: dr. sc.  / doc. dr. sc. / izv. prof. dr. sc. / prof. dr. sc. 
% mentor's title:
% EN: -prazno- / Asst. Prof.  / Assoc. Prof.       / Full Prof.

\godina{2024}
\mjesec{rujan}
% mjesec obrane rada ili projekta
% year and month of the presentation of the project or paper

\indeks{35918/07–R}
% broj indeksa ili JMBAG
% author's ID

\smjer{Informacijski i poslovni sustavi}
% (ili:
%     Informacijski sustavi, 
%     Poslovni sustavi, 
%     Ekonomika poduzetništva, 
%     Primjena informacijske tehnologije u poslovanju, 
%     Informacijsko i programsko inženjerstvo, 
%     Baze podataka i baze znanja, 
%     Organizacija poslovnih sustava, 
%     Informatika u obrazovanju
% )
% study programme; please enter "Erasmus" for incoming exchange students


\sazetak{Opsega od 100 do 300 riječi. Sažetak upućuje na temu rada, ukratko se iznosi čime se rad bavi, teorijsko-metodološka polazišta, glavne teze i smjer rada te zaključci.}
% abstract of 100 to 300 words.

\kljucneRijeci{riječ; riječ; ...riječ; Obuhvaća $7\pm2$ ključna pojma koji su glavni predmet rasprave u radu.}
% keywords including 7 +/- 2 syntagms

\acrodef{VAS}{višeagentni sustav}


\begin{document}

\maketitle

\tableofcontents

\makeatletter \def\@dotsep{4.5} \makeatother
\pagestyle{plain}



\chapter{Uvod}

Završni ili diplomski rad studenta/studentice je konačni rezultat uloženog napora u završetak studija. Obranom završnog ili diplomskog rada student/studentica stječe prava i obveze koje proizlaze iz završetka akademskog obrazovanja. S ciljem osiguranja potpore studentima pri pisanju završnog/diplomskog rada, izrađen je ovaj predložak oblikovanja samog rada.

Načelna napomena o strukturi rada jest da se nazivi i struktura poglavlja obavezno definiraju u dogovoru s mentorom/mentoricom. Sadržajna preporuka je da u uvodu treba opisati što je tema završnog/diplomskog rada, zašto je tema značajna te koja je motivacija studenta/ studentice za odabir teme \cite{oraictolic2011AkademskoPismoStrategije}.



\chapter{Metode i tehnike rada}

U ovom poglavlju treba opisati koje će metode i tehnike biti korištene pri razradi teme, kako su provedene istraživačke aktivnosti, koji su programski alati ili aplikacije korišteni.

\lipsum[1-2]



\chapter{Razrada teme}

Ovo je glavni dio rada u kojem treba razraditi temu, pojasniti istraživanja, prikazati rezultate i slično. Poželjno je na početku poglavlja dati kratki opis strukture poglavlja, kako bi čitatelj/čitateljica rada mogao/mogla lakše pratiti složenu cjelinu.


\chapter{Zaključak}

Ovdje treba sažeto rezimirati najvažnije rezultate razrade teme rada. Potrebno je sažeto opisati što je predmet rada \cite{copeland2020ArtificialIntelligence}, koje su metode, tehnike, programski alati ili aplikacije korištene u razradi rada te koje su pretpostavke dokazane, a koje opovrgnute. Sadržajno, ono što se u uvodu rada najavljuje i kasnije je obuhvaćeno u samom radu, moralo bi biti opisano u zaključnom dijelu kroz rezultate rada. 

\lipsum[1-2]

\makebackmatter
% generira popis korištene literature, popis slika (ako je primjenjivo), popis tablica (ako je primjenjivo) i popis isječaka koda (ako je primjenjivo)

\appendices % ako nije potrebno, obrisati ili zakomentirati

\chapter{Prilog 1} % ako nije potrebno, obrisati ili zakomentirati

\chapter{Prilog 2} % ako nije potrebno, obrisati ili zakomentirati

\end{document}
